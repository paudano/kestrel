% Copyright (c) 2017 Peter A. Audano III
% GNU Free Documentation License Version 1.3 or later
% See the file COPYING.DOC for copying conditions.

\section{Supplementary Information}
\label{sec.suppl}


%%%%%%%%%%%%%%%%%%%%%%%%%%%%%%%%%
%%% Command Line Return Codes %%%
%%%%%%%%%%%%%%%%%%%%%%%%%%%%%%%%%
\subsection{Command Line Return Codes}
\label{sec.supl.retcode}

The KAnalyze user interface always returns a well-defined code. When executed from a script environment, this return code can easily be checked to see if KAnalyze completed normally or not. Each return code is defined in this section.

In the following list of return codes, the numeric code is listed. For API users, the constant defined in \texttt{edu.gatech.kanalyze.Constants} is also
listed.

\begin{description}
\item[0 (ERR\_NONE)] The program terminated normally. All k-mers were successfully processed without error, or the help option was invoked.

\item[1 (ERR\_USAGE)] Command line arguments were incomplete, improperly formatted, or required arguments were missing.

\item[2 (ERR\_IO)] An I/O (input/output) error occurred reading or writing data. This error is normally returned for file I/O errors.

\item[3 ERR\_SECURITY] A security error, such as permissions denied, occurred.

\item[4 (ERR\_FILENOTFOUND)] A required or specified file was not found.

\item[5 (ERR\_DATAFORMAT)] Data in an input file is improperly formatted. 

\item[6 (ERR\_ANALYSIS)] Some un-recoverable error occurred during analysis.

\item[7 (ERR\_INTERRUPTED)] A program thread was interrupted while it was running. This would normally be returned if the program is terminated before it completes.

\item[8 (ERR\_LIMITS)] A limitation was reached that forced the program to terminate. This should only occur in extreme cases and may indicate an problem with the input. For example, if more than $2^{63}$ sequences are in a FASTA file, it will trigger this error.

\item[98 (ERR\_ABORT)] The program or a process was terminated before it completed. When the action is requested or expected, this should be returned in lieu of ERR\_INTERRUPTED even if interrupting threads was necessary.

\item[99 (ERR\_SYSTEM)] Some serious unrecoverable system error occurred. These errors are almost certainly program bugs. Some other unrecoverable errors, such as running out of memory, could return this code.

\end{description}


%%%%%%%%%%%%%%%%%%%%%%%%%%%%
%%% Building From Source %%%
%%%%%%%%%%%%%%%%%%%%%%%%%%%%
\subsection{Building From Source}
\label{sec.suppl.building}

The source is distributed with an Apache Ant build file with multiple targets. To build these, Apache Ant must be installed.

To run a target, enter \texttt{ant <target>}. Multiple targets may be supplied, for example \texttt{ant clean package}.

The most commonly used targets are ``clean'', ``compile'', and ``package''. The clean removes all temporary files. It is not often necessary, but it should be executed before building a package for testing to ensure no artifacts are left behind from previous builds. The compile target compiles the class files and does nothing else. The package target generates the JAR file and distributable packages.

The ``doc.javadoc'' target generates the Javadoc pages from Kestrel source comments. See Section~\ref{sec.suppl.building.javadoc} for more information about these pages. ``doc.manual'' generates this manual as a PDF file.

Other targets are called by the targets described above and are not often called directly.

For a full list of targets and brief descriptions, enter \texttt{ant -projecthelp}.


%%%%%%%%%%%%%%%%%%%%%%%%%%%%%%
%%% Building Javadoc Pages %%%
%%%%%%%%%%%%%%%%%%%%%%%%%%%%%%
\subsubsection{Building Javadoc Pages}
\label{sec.suppl.building.javadoc}

Javadoc is a very powerful tool for documenting APIs written in Java. Section~\ref{sec.api.doc} introduces the concept and the KAnalyze rules governing these comments.

Javadoc comments begin with ``/**'', and end with ``*/''. In Kestrel, these comments appear before every class, method, and field. The comment starts with a description of what the element does. For methods, it documents parameters and the conditions under which all exceptions are thrown. Full documentation for writing these comments can be found online\footnote{http://www.oracle.com/technetwork/java/javase/documentation/index-137868.html}. 

The ant build system (Section~\ref{sec.suppl.building}) generates web pages from these comments. The ``doc.javadoc'' target builds two sets of pages. One, the API documentation, documents all elements available on the API. This target is intended for developers who are extending or using the API. The other, FULL documentation, documents everything including private members that are not available to the API. This is meant for Kestrel maintenance programmers.

After running the javadoc Ant target, the API documentation can be access by loading ``build/doc/javadoc/api/index.html'' in a browser (relative to the project root). The FULL documentation can be access by loading ``build/doc/javadoc/full/index.html'' in a browser (relative to the project root).
